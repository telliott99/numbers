\documentclass[11pt, oneside]{article} 
\usepackage{geometry}
\geometry{letterpaper} 
\usepackage{graphicx}
	
\usepackage{amssymb}
\usepackage{amsmath}
\usepackage{parskip}
\usepackage{color}
\usepackage{hyperref}

\graphicspath{{/Users/telliott/Github/figures/}}
% \begin{center} \includegraphics [scale=0.4] {gauss3.png} \end{center}

\title{Natural Numbers}
\date{}

\begin{document}
\maketitle
\Large

%[my-super-duper-separator]
\section*{Integers}

The \emph{natural} or counting numbers which everyone learns very early in life are $1, 2, 3$ and so on.

One can get hung up on the question of whether the natural numbers would exist without the problem of counting a dozen sheep or all twenty of our fingers and toes.  Leopold Kronecker famously said "God made the integers; all else is man's handiwork".

We will not worry about where they come from.

Mathematicians refer to the \emph{set} of natural numbers and give that set a special symbol, $\mathbb{N}$.  We write
\[ \mathbb{N} = \{ 1, 2, 3 \dots \} \]

The brackets contain between them the elements or members of the set. The dots mean that this sequence continues forever.

How can we decide whether a particular $n$ is in the set if we can't enumerate all of its members?  We can tell by its form whether some $n$ is a natural number or not.  

If this seems problematic, you might call $\mathbb{N}$ a class instead (Hamming);  we carry out \emph{classification} to decide whether $n$ is a natural number.

The notion of an unending sequence can be unnerving upon first encounter.

\subsection*{construction of N}

To construct the set $\mathbb{N}$, start with the smallest element, $1$.  Then 
\[ 1 + 1 = 2 \]
\[ 2 + 1 = 3 \]
\[ 3 + 1 = 4 \]
\[ \dots \]
Add successive elements by forming $a_n + 1 = a_{n+1}$.

$\mathbb{N}$ is an infinite set.

We say there is no largest number in $\mathbb{N}$, no largest $n \in \mathbb{N}$.  The symbol $\in$ means "in the set" or "is a member of the set".

Proof:  

Suppose $\mathbb{N}$ did have a largest member, $M$.  

Well, what about $M + 1$?  By the definition we can construct it and it is clearly a member of the set, but $M + 1 > M$ so $M$ is not the largest number in the set.

This is a proof by contradiction that $\mathbb{N}$ is infinite.

$\square$

\subsection*{set membership}

Sometimes people say that
\[ 0 \in \mathbb{N} \]
(0 is a part of the set) but most do not, and we will follow the definition given above.  If you wanted to be explicit about this you could write
\[ 0 \notin \mathbb{N} \]

What do we mean by infinity?  We mean an upper bound on the natural numbers, and later, all rational and indeed all real numbers.  

All numbers $n \in \mathbb{N}$ have the property that $n$ is contained in the interval $[1..\infty)$.  However, $\infty$ is \emph{not} considered part of the interval, and that is the meaning of the the right parenthesis.

$\infty$ is not a number so it probably doesn't even make sense to write $\infty \notin \mathbb{N}$.

\subsection*{least element}

$\mathbb{N}$ does not have a greatest number, but it does have a smallest or least one.  If pairwise comparisons are carried out, a single element, the number $1$, has the property that $1 \le n$ for all numbers $n \in \mathbb{N}$.  As we go on, we will find that other types of numbers (rationals and real numbers), do not have a least positive number.

\subsection*{well-ordered property}

Since we can also find the least member of the set excluding $1$, written $\mathbb{N} \setminus 1$, we can order every number in $\mathbb{N}$.  

This property is called the \textbf{well-ordered} property.

\subsection*{the Integers}

The set $\mathbb{Z}$ contains all the members of $\mathbb{N}$ plus their negatives, as well as the special number $0$, often called the additive identity since $0 + n = n$ for all $n \in \mathbb{N}$.

\[ \mathbb{Z} = \{ \dots -2, -1, 0, 1, 2, \dots \} \]

$\mathbb{Z}$ stands for the German word \emph{Zahlen}, number.  The set $\mathbb{Z}$ is usually referred to as the integers.

$\mathbb{Z}$ is also an infinite set and also has the well-ordered property.  To show this simply order all numbers $n > 0$ with respect to zero using $<$, and all the numbers $n < 0$ using $>$.


\end{document}